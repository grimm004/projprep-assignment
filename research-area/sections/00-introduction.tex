Location-based services (LBSs)~\cite{schiller2004location, junglas2008location, dey2009location, kupper2005location} is a field which spans a large array of disciplines and technologies, such services are increasingly used in industry and by people in many differing forms~\cite{zickuhr2013location}.

Recently, they have seen increased use in businesses with high traffic flows of people and customers.
Some examples of these include shops, cruise liners, stadiums and airports.

A use-case for LBSs in these scenarios is occupancy sensing~\cite{shen2020wi} and finding customer flow hot-spots.
This enables businesses to determine how much value to assign given areas within their confines (and thus how much to charge for leasing and where to place advertisements).
Additionally, the services may be used to directly benefit end-customers by providing location-based customer service (for example delivering food or drinks directly to the customer anywhere on the premises).

Another application for LBSs is in assisting with safety and security.
Given the modern example of Covid-19, LBSs can detect clusters of people and inform the actions necessary to disperse them.

The data these services are based on is typically collected either through dedicated nodes placed at known locations sending and receiving probe requests or trivially by any WiFi hotspots the devices are already connected to.
